\documentclass[final, 12pt, aspectratio=169, xcolor={dvipsnames}]{beamer}
\usepackage{lmodern}
\usepackage[utf8]{inputenc}
\usepackage[T1]{fontenc}
\usepackage{graphicx} 
\usepackage{textpos} % package for the positioning
%\usepackage{xcolor}
\usepackage{tcolorbox}
\usepackage{tabularx}
\usepackage{hanging}
\usepackage{animate}
\usepackage{caption}
%\usepackage{grffile}

\title[PEARL]{The demographic projections of the Netherlands Environmental Assessment Agency (PBL) and Statistics Netherlands (CBS)}
\subtitle[PEARL]{}
\author[T. Husby]{Dr. Trond Grytli Husby\ }
\institute[PBL]{
  Netherlands Environmental Assessment Agency (PBL) \\[5ex]
  \texttt{trond.husby@pbl.nl}
}
\date[\today]{Amsterdam 16 September, 2020}

% set path to figures
\newcommand*{\figs}{../figs}%

% position the logo
\addtobeamertemplate{background}{}{%
  %\begin{textblock*}{100mm}(0.95\textwidth, 7cm)
  %\begin{flushright}
    \includegraphics[width=\paperwidth, height=\paperheight]{\figs/pbl_background.pdf}
  %\end{flushright}
    %\end{textblock*}}
}

% colour scheme and settings
\setbeamercolor{title}{bg=white,fg=blue!35!black}
\setbeamercolor{frametitle}{bg=,fg=PineGreen}
\setbeamercolor{enumerate item}{fg=PineGreen}
\setbeamercolor{itemize item}{fg=PineGreen}
\setbeamertemplate{itemize item}[circle]
\setbeamercolor{itemize subitem}{fg=PineGreen}
\setbeamertemplate{itemize subitem}[triangle]
\setbeamerfont{frametitle}{size=\normalsize}
\addtobeamertemplate{frametitle}{\vspace*{1cm}}{\vspace*{0.0cm}}
\setbeamertemplate{footnote}{\hangpara{2em}{1}\makebox[2em][l]{\insertfootnotemark}\footnotesize\insertfootnotetext\par}

% miscellaneous
\newcommand{\semitransp}[2][35]{\color{fg!#1}#2}
\newcommand{\source}[1]{\caption*{\tiny Source: {#1}} }


\begin{document}

\beamertemplatenavigationsymbolsempty

%--- the titlepage frame -------------------------%
{
  \setbeamertemplate{footline}{}

  \begin{frame}
    \titlepage
  \end{frame}
}


%---  --------------------------------%

\begin{frame}{Plan for the day}  
  \begin{enumerate}
  \item Population projections: definition, relevance
  \item Uncertainty and scenarios
  \item Projection of trend
  \item The cohort-component model and growth components
  \item The Regional population projections by CBS/PBL 2019 \href{https://themasites.pbl.nl/regionale-bevolkingsprognose/}{\beamergotobutton{Link}}
  \item Accuracy of previous projections
    \end{enumerate}
\end{frame}


%---  --------------------------------%
\begin{frame}{Who am I}
  \begin{itemize}
  \item Norwegian, living in the Netherlands for 10 years 
  \item PBL for 4 years
  \item Working with the population projections for 2 year
    \item Economist by training (VU Amsterdam)
    \end{itemize}
\end{frame}

%---  --------------------------------%  

\begin{frame}{Why am I working with this}
  \noindent
  \begin{minipage}[t]{0.48\linewidth}%
    \begin{itemize}
    \item It's my job...
    \item Population change and projections thereof are important for policy, for example on the housing market
    \item The underlying drivers of population changes are, at least to a researcher, interesting in themselves
    \end{itemize}
\end{minipage}%
\hfill%
\begin{minipage}[t]{0.48\linewidth}
%  \vspace{-1cm}
  \begin{figure}
    \includegraphics[scale = 0.2]{\figs/witch.png}
    \source{\url{https://img.clipartxtras.com/03843abee543e741be870c4ada22c760_free-to-use-public-domain-cauldron-clip-art-witch-cauldron-clipart_500-500.png}}
  \end{figure}  
\end{minipage}    
  
\end{frame}


%--- Projections general --------------------------------%  
\begin{frame}{Definition of projections}
  \begin{itemize}
  \item  Van Dale: \textit{statement about the probable course or the probable outcome or outcome of elections, competitions and the like.}
  \item Projections $\neq$ forecast $\neq$ scenario studies
  \item Projections are, almost by definition, wrong: population projections with no error are most likely result of coincidence 
  \item Think of projections as an attempt of establishing the \textit{most probable} trajectory of population change 
    \item Anyone working with population projections should have this in mind
  \end{itemize}
\end{frame}

\begin{frame}{Why do we make demographic projections?}
  \begin{itemize}
  \item Population change is important in other settings: house prices and rent in Amsterdam are high because many people want to live in Amsterdam 
  \item Policy makers use them: plans for new residential areas, schools, location of hospitals...
    \item Economists use them: gdp growth = population growth + productivity growth
  \item Important: normative (is population growth good or bad?) versus descriptive (what are the main drivers of population growth?) analysis 
  \end{itemize}
\end{frame}

\begin{frame}{Why do we make (regional) demographic projections?}
    \begin{minipage}[t]{0.48\linewidth}%
      %\vspace{-1cm}
      \begin{itemize}
      \item National projections: population in the Netherlands is projected to grow
      \item Population growth is not uniformly distributed: which regions will grow and which will decline?
      \item What is the regional impact on population of trends in the components? Will the trends continue into the future (subjective)?
  \end{itemize}
 
\end{minipage}%
\hfill%
\begin{minipage}[t]{0.48\linewidth}
%  \vspace{-1cm}
       \begin{figure}
        \includegraphics[trim={50cm 7cm 10cm 5cm},clip, angle = -90, scale = 0.05]{\figs/20181123_105808.jpg}        
      \end{figure}

\end{minipage}  
\end{frame}

\begin{frame}{Why do we make demographic projections?}
  \begin{minipage}[t]{0.48\linewidth}%
    %\vspace{-1cm}
    \begin{figure}
      \includegraphics[scale = 0.18]{\figs/coventry_green.jpg}
      \source{\url{https://www.coventrytelegraph.net/news/coventry-news/revealed-swathes-green-belt-lost-10674329}}
    \end{figure}
\end{minipage}%
\hfill%
\begin{minipage}[t]{0.48\linewidth}
  \small
  Based on population projections from the UK ONS, the city council of Coventry is planning to develop its Green Belt. Not everyone is happy \\
  \begin{quotation}
    Population projections might have some
value in projecting school numbers five years ahead (although recent experience in the Council
might cast doubt on that) but projecting twenty years up to 2031 is completely unrealistic. \href{http://www.coventrysociety.org.uk/js/plugins/filemanager/files/information/Coventry_Local_Plan_2014_Response_-_final_with_corrections.pdf}{\beamergotobutton{Link}}     
    \end{quotation}
  
\end{minipage}
  
\end{frame}

\begin{frame}{Demography and projections}
  \begin{itemize}
  \item Demography: description of the past 
  \item Projections: best guess of the future
  \item Population and households:
    \begin{itemize}
    \item \textit{Amount}: number of households and people 
    \item \textit{Composition}: distribution by gender, age, ethnicity/background, household position 
    \item \textit{Spatial distribution}: world, country, region (province, COROP, municipality, neighbourhood)
      \item \textit{Changes}: birth, death, immigration, emigration, departure or arrival, transitions between household positions
    \end{itemize}
    \item Cohort-component model: population change from trends in growth components
  \end{itemize}
\end{frame}

\begin{frame}{Cohort-component model}
  \begin{minipage}[t]{0.48\linewidth}%
    \begin{enumerate}
    \item Input: initial population (beginning of the year)
    \item Calculate number of deaths by cohort
    \item Calculate migration (gross or net)
    \item Calculate births by cohort
    \item Output: projected population (end of the year)
      \item Same procedure next year, until you reach target year
    \end{enumerate}   
\end{minipage}%
\hfill%
\begin{minipage}[t]{0.48\linewidth}
  \begin{figure}
    \includegraphics[scale = 0.3]{\figs/smith_cohort_component_model.png}
    \source{Smith et al. (2013). A Practitioner's Guide to State and Local Population Projections}
  \end{figure}
\end{minipage}%
\end{frame}

\begin{frame}{The origins and definition of uncertainty}
  \begin{itemize}
  \item Typology: Unknown knowns; known unknowns; unknown unknowns.
    \item Population projections only account for the first type
  \item The cohort-component model is a book-keeping system; per definition no uncertainty
  \item However, substantial uncertainty in the trend of the growth components
    \item Incorporating uncertainty: create projection interval with possible outcomes (according to the model) 
    \end{itemize}
\end{frame}

\begin{frame}{Uncertainty in the projections}
  \begin{itemize}
    \item Circularity: what you are trying to project might be affected by your projection
    \item Self-fulfilling prophecy: building plans reflect projections, projection becomes true
    \item ...or self-denying prophecy: policy makers do not like projection and restrict migration, projection is wrong
    \end{itemize}
\end{frame}

\begin{frame}{Scenarios versus projection}
  \begin{minipage}[t]{0.48\linewidth}%
    \begin{itemize}
    \item Scenario's are quantification of future \textit{story lines}, not necessarily the \textit{most likely} outcomes
    \item A way to structure discussion of possible futures, for example about the impacts of climate change
    \end{itemize}   
\end{minipage}%
\hfill%
\begin{minipage}[t]{0.48\linewidth}
  \vspace{-1cm}
  \centering
  \begin{figure}
    \includegraphics[scale = 0.4]{\figs/ssp_scenarios.png}
    \source{\url{https://climate4impact.eu/files/SSPs.png}}
    \end{figure}
\end{minipage}

\end{frame}

\begin{frame}{Projecting growth components: requirements}
  In order to make a \textit{quantitative} projection, the following two conditions need to be satisfied: 
  \begin{enumerate}
  \item Historic data of what we are projecting is available
    \item We can reasonably assume that patterns and relationships from the past will (at least to some degree) extend into the future
    \end{enumerate}
\end{frame}

\begin{frame}{Trend change: break down or break dance?}
  \begin{minipage}[t]{0.48\linewidth}%
    The fraction of within-municipality moves for some municipalities hosting an asylum-seeker centre. What is the fraction in 2030?
\end{minipage}%
\hfill%
\begin{minipage}[t]{0.48\linewidth}
  \vspace{-1cm}
  \centering
  \includegraphics[scale = 0.4]{\figs/within_mun_fraction.png}    
\end{minipage}    
\end{frame}

\begin{frame}{Trend change: break down or break dance?}
  \begin{minipage}[t]{0.48\linewidth}%
    {\semitransp{The fraction of within-municipality moves for some municipalities hosting an asylum-seeker centre. What is the fraction in 2030?}} Relatively small municipalities where a large proportion moved from another municipality and moved quickly to another municipality.
\end{minipage}%
\hfill%
\begin{minipage}[t]{0.48\linewidth}
  \vspace{-1cm}
  \centering
  \includegraphics[scale = 0.4]{\figs/within_mun_fraction.png}
\end{minipage}    
\end{frame}


\begin{frame}{Trend change: break down or break dance?}
  \begin{minipage}[t]{0.48\linewidth}%
    \begin{itemize}
    \item In the upper panel, the frequency of internal migration: number of (in this case, monthly) moves per 1000 inhabitants
    \item What is the fraction in 2030?
    \item Time series is moving up and down with irregular intervals, correlating with the business cycle
      \item We do not know the state of the business cycle in 2030, an average (around 8.5) might be a good guess
      \end{itemize}
      
\end{minipage}%
\hfill%
\begin{minipage}[t]{0.48\linewidth}
  \vspace{-0.2cm}
  \centering
  \includegraphics[scale = 0.45]{\figs/freq--freq-plot-1.pdf}    
\end{minipage}    

\end{frame}

\begin{frame}{End of part 1}
  See you in a sec!
\end{frame}


%--- Second half --------------------------------%

\begin{frame}{Cohort-component model (national)}
  
    $P_{t+1} = P_{t} + B - D + I - $ \textit{E}
    
  \begin{description}
  \item \textit{$P_{t}$}: population in $t$ \\
  \item \textit{B}: births in the interval $(t, t+1 )$ \\
  \item \textit{D}: deaths in the interval $(t, t+1 )$ \\
  \item \textit{I}: immigration in the interval $(t, t+1 )$ \\
  \item \textit{E}: emigration in the interval $(t, t+1 )$
  \end{description}
\vspace{1cm}
  X typically calculated as $\frac{X_{t} + X_{t+1}}{2}$
  
\end{frame}

\begin{frame}{Growth components}
  \begin{minipage}[t]{0.48\linewidth}%
    Cross-sectional (calendar year)
    \begin{itemize}
    \item Changes related to the business cycle
    \item Fluctuations in birth rates
      \item Stricter immigration policies
    \item Enlargement of the EU (work-related migration)
      \item War (asylum seekers)
      \end{itemize}
 
\end{minipage}%
\hfill%
\begin{minipage}[t]{0.48\linewidth}
  Longitudinal (cohort)
  \begin{itemize}
  \item Structural changes
  \item Average number of children from 3 to 2
  \item People live longer
      \end{itemize}
\end{minipage}    
\end{frame}

\begin{frame}{Growth components at different scales}
  \noindent
\begin{minipage}[t]{0.48\linewidth}%

  World: births and deaths \\
  \semitransp{National: births, deaths, net migration} \\
  \semitransp{Regional: births, deaths, net migration, arrivals and departures} \\
  
\end{minipage}%
\hfill%
\begin{minipage}[t]{0.48\linewidth}
  \vspace{-1cm}
  \begin{figure}
    \includegraphics[scale = 0.1]{\figs/Planet_Earth_8.jpg}
    \source{\url{http://4.bp.blogspot.com/-8pNWCR1SJhY/T83V8A2HzXI/AAAAAAAADuM/aRub2XdIBEo/s1600/Planet+Earth+8.jpg}}
  \end{figure}
\end{minipage}    

\end{frame}

\begin{frame}{Growth components at different scales}
  \noindent
\begin{minipage}[t]{0.48\linewidth}%

  {\semitransp{World: births and deaths}} \\
  {National: births, deaths, net migration} \\
  \semitransp{Regional: births, deaths, net migration, arrivals and departures} \\
  
\end{minipage}%
\hfill%
\begin{minipage}[t]{0.48\linewidth}
  \vspace{-1cm}
  \begin{figure}
    \includegraphics[scale = 0.1]{\figs/migration_flows.png}
    \source{\url{http://www.slate.com/content/dam/slate/blogs/the_world_/2014/04/02/world_on_the_move_five_years_of_global_migration_in_one_chart/circular_plot_flows_between_world_regions_200510_1.png.CROP.cq5dam_web_1280_1280_png.png}}
  \end{figure}
\end{minipage}    
  
\end{frame}

\begin{frame}{Growth components at different scales}
  \noindent
\begin{minipage}[t]{0.48\linewidth}%

  {\semitransp{World: births and deaths}} \\
  {\semitransp{National: births, deaths, net migration}} \\
  {Regional: births, deaths, net migration, arrivals and departures} \\
  
\end{minipage}%
\hfill%
\begin{minipage}[t]{0.48\linewidth}
  %\vspace{-1cm}
  \begin{figure}
    \includegraphics[scale = 0.09]{\figs/touw_blok_adam.jpeg}
    \source{\url{https://i.ytimg.com/vi/68nvHI10BdA/maxresdefault.jpg}}
  \end{figure}
\end{minipage}    
  
\end{frame}

\begin{frame}{Regional population projections}
  \begin{minipage}[t]{0.48\linewidth}%
    Top-down: disaggregate national numbers to the region \\
    +: straight forward to implement \\
    +: per definition consistent with national numbers \\
    --: little/no insight into determinants of regional population change
 
\end{minipage}%
\hfill%
\begin{minipage}[t]{0.48\linewidth}
  Bottom-up: calculate regional numbers consistent with national numbers \\

  +: insights into determinants of regional population change \\
  --: heavy data and time requirements \\
  --: extra step required to ensure consistency with national numbers \\
\end{minipage}    
\end{frame}

%--- Regional CBS/PBL projections --------------------------------%
\begin{frame}{Regional population projections}
  \begin{itemize}
  \item  Projections of population, households and demographic events in Dutch municipalities until 2040
  \item Carried out every three years: most recent edition published last year!
  \item Regional projections (PBL/CBS) are made consistent with the national projections (CBS)
  \item The projections made with the cohort-component model PEARL (Projecting population Events at Regional Level)
    \item Trends in components of population growth and transition rates between household positions are projected separately as inputs to the model
    \end{itemize}
\end{frame}

\begin{frame}{Growth next to decline, population and households}
  \begin{minipage}[t]{0.48\linewidth}%
    \begin{figure}
      \includegraphics[trim={0 0 4cm 0cm},clip, scale = 0.65]{\figs/003k_rpg19_Bevolkingsontwikkeling_per_gemeente_2018_2035.png}
    \end{figure}
  \end{minipage}
  \begin{minipage}[t]{0.48\linewidth}
    \begin{figure}
      \includegraphics[trim={0 0 0 0cm},clip, scale = 0.65]{\figs/017k_rpg19_Ontwikkeling_aantal_huishoudens_per_gemeente_2018_2035.png}
  \end{figure}
    \end{minipage}
\end{frame}



\begin{frame}{Total fertility rate}
  \begin{itemize}
  \item TFR = average number of children per woman for each municipality
  \item Regional variation due to
    \begin{itemize}
    \item share of singles among women in age group 20 - 40 (-)
    \item share of votes for Christian parties (+)
    \item share of people on social benefits (-)
      \item building construction (+)
    \end{itemize}
    \item Changes in the share of single women and building construction affect projected TFR
    \end{itemize}
  
  \end{frame}

\begin{frame}{Immigration and emigration}
  \begin{itemize}
  \item Since 2015 asylum migration has declined, but total migration still grows. Primarily from EU countries, primarily for work, study or reuniting with family
  \item Emigration often follows immigration in size (many return), but this time to a lesser extent
  \item Assumption: the current high net migration is a cycle-effect, immigration will decrease and emigration will increase
  \item Regional: concentration index of immigration and emigration
    \item $CI_{r} = \frac{immigrants_{r} / \sum_{r} immigrants_{r}}{population_{r} / \sum_{r} population_{r}}$
    \end{itemize}
  \end{frame}

\begin{frame}{Immigration and emigration}
  \noindent
  \begin{minipage}[t]{0.48\linewidth}%
    \begin{figure}
    \includegraphics[scale = 0.4]{\figs/immigration_2016_P2019.png}
    %\includegraphics[scale = 0.09]{\figs/{tfr_2035_P2019.png}}
  \end{figure}

\end{minipage}%
\hfill%
\begin{minipage}[t]{0.48\linewidth}
  %\vspace{-1cm}
  \begin{figure}
    \includegraphics[scale = 0.4]{\figs/emigration_2016_P2019.png}
    %\includegraphics[scale = 0.09]{\figs/{tfr_2035_P2019.png}}
  \end{figure}
\end{minipage}
  
  \end{frame}

\begin{frame}{Short-distance migration and the dwelling stock}
\begin{itemize}
    \item PEARL distinguishes long- and short distance migration (cut-off 35 km)
      \item Long distance migration probability matrix extrapolated from past observations
      \item Distribution of short-distance migration predicted with a spatial interaction model, with population at destination, distance between origin and destination and building construction at destination as explanatory variables.
      \end{itemize}
\end{frame}

\begin{frame}{Short-distance migration and the dwelling stock}
  \begin{itemize}
  \item Strong relation between population growth and the number of dwellings: effect of new construction is particularly strong in municipalities with a shortage of housing
  \item Short-distance migration in PEARL:
    \begin{enumerate}
    \item Potential moves estimated with spatial interaction model
    \item Actual migration calculated by taking into account housing availability (potentially moving back to 1…)
    \end{enumerate}
  \item Building plans are obtained through a combination of desk research and consultation with municipalities and provinces
  \end{itemize}
  \href{http://www.pbl.nl/publicaties/pbl-cbs-regionale-bevolkings-en-huishoudensprognose-2016-2040-woningbouwveronderstellingen}{\beamergotobutton{Link}}
\end{frame}

\begin{frame}{How accurate were previous projections? \footnote{\tiny Population in 2017 according to the 2013 edition, \\ compared with the data.} \footnote{\tiny $error = \frac{actual - prediction}{actual} * 100$}}
    \centering
    \includegraphics[width = 0.4\textwidth, height = 0.7\textheight]{\figs/regionale_prognose_accuracy.png} 

\end{frame}

\begin{frame}{How accurate were previous projections?\footnote{\tiny Large: >100000 inhabitants}}
  \centering
  \includegraphics[width = 0.65\textwidth, height = 0.7\textheight]{\figs/regionale_prognose_accuracy_split.png}

\end{frame}

\begin{frame}{Explaining deviances}  
  \begin{itemize}
  \item  2012 was, in terms of the housing market, still very much a crisis year
  \item Plans for future building by municipalities and provinces were likely pessimistic
  \item The sour mood worked through the projections of short-distance migration into the population projections!
    \item The deviations were larger in small municipalities (still PEARL did better than a naive forecast)
  \end{itemize}
  \href{http://www.pbl.nl/publicaties/evaluatie-pbl-cbs-regionale-bevolkings-en-huishoudensprognose}{\beamergotobutton{Link}}
\end{frame}

\begin{frame}{Questions}
  \begin{itemize}
  \item So...projections are, almost by definition, wrong. Why do we bother?
  \item Municipalities and provinces base their building plans on them, should they? \href{https://www.gebiedsontwikkeling.nu/artikelen/het-noord-hollandse-sprookje-van-vraaggestuurde-woningbouw/}{\beamergotobutton{Link}}
  \item An alternative is judgmental forecast: project developer decides what/where/how much should be built.
  \item What can possibly go wrong here...?
  \end{itemize}
  \end{frame}

\begin{frame}{Conclusions}  
  \begin{itemize}
  \item \textit{Prediction is difficult, especially about the future} Niels Bohr (?)
  \item \textit{Essentially all models are wrong, but some are useful} George Box
  \item Uncertainty is inherent in any projection, and projections are almost never exactly accurate.
  \item Population projections are nevertheless useful for policy making. And it is not clear whether the alternative (judgmental forecast) is better.
  \item Regional population projections of CBS/PBL: check out the link of page 2!

    \end{itemize}
\end{frame}


\end{document}


